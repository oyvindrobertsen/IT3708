%%%%%%%%%%%%%%%%%%%%%%%%%%%%%%%%%%%%%%%%%
% Journal Article
% LaTeX Template
% Version 1.3 (9/9/13)
%
% This template has been downloaded from:
% http://www.LaTeXTemplates.com
%
% Original author:
% Frits Wenneker (http://www.howtotex.com)
%
% License:
% CC BY-NC-SA 3.0 (http://creativecommons.org/licenses/by-nc-sa/3.0/)
%
%%%%%%%%%%%%%%%%%%%%%%%%%%%%%%%%%%%%%%%%%

%----------------------------------------------------------------------------------------
%	PACKAGES AND OTHER DOCUMENT CONFIGURATIONS
%----------------------------------------------------------------------------------------

\documentclass[twoside]{article}

\usepackage{lipsum} % Package to generate dummy text throughout this template


\usepackage[sc]{mathpazo} % Use the Palatino font
\usepackage[T1]{fontenc} % Use 8-bit encoding that has 256 glyphs
\usepackage[utf8]{inputenc}
\linespread{1.05} % Line spacing - Palatino needs more space between lines
\usepackage{microtype} % Slightly tweak font spacing for aesthetics
\usepackage{amsmath}
\usepackage{listings}

\lstset{
    basicstyle=\ttfamily\footnotesize,
    breaklines=true
}


%\usepackage[hmarginratio=1:1,top=32mm,columnsep=20pt]{geometry} % Document margins
\usepackage[margin={1cm,2cm}]{geometry}
\setlength{\columnsep}{1cm}
\usepackage{multicol} % Used for the two-column layout of the document
\usepackage[hang, small,labelfont=bf,up,textfont=it,up]{caption} % Custom captions under/above floats in tables or figures
\usepackage{booktabs} % Horizontal rules in tables
\usepackage{float} % Required for tables and figures in the multi-column environment - they need to be placed in specific locations with the [H] (e.g. \begin{table}[H])
\usepackage{hyperref} % For hyperlinks in the PDF
\usepackage{multirow}

\usepackage{lettrine} % The lettrine is the first enlarged letter at the beginning of the text
\usepackage{paralist} % Used for the compactitem environment which makes bullet points with less space between them

\usepackage{abstract} % Allows abstract customization
\renewcommand{\abstractnamefont}{\normalfont\bfseries} % Set the "Abstract" text to bold
\renewcommand{\abstracttextfont}{\normalfont\small\itshape} % Set the abstract itself to small italic text


\usepackage{graphicx}

\usepackage{tikz}


\newcommand{\rparen}{)}

\usepackage{titlesec} % Allows customization of titles
\renewcommand\thesection{\Roman{section}} % Roman numerals for the sections
%\renewcommand{\thesubsection}{\thesection\hspace{1mm}\alph{subsection}}
\titleformat{\section}[block]{\large\scshape\centering}{\thesection}{1em}{} % Change the look of the section titles
\titleformat{\subsection}[block]{\large}{\thesubsection}{1em}{} % Change the look of the section titles

\usepackage{fancyhdr} % Headers and footers
\pagestyle{fancy} % All pages have headers and footers
\fancyhead{} % Blank out the default header
\fancyfoot{} % Blank out the default footer
\fancyhead[C]{IT3708 Sub-symbolic AI Methods $\bullet$ Project 3 $\bullet$ \date{\today}} % Custom header text
\fancyfoot[RO,LE]{\thepage} % Custom footer text

%----------------------------------------------------------------------------------------
%	TITLE SECTION
%----------------------------------------------------------------------------------------

\title{\vspace{-15mm}\fontsize{18pt}{10pt}\selectfont\textbf{Evolving Neural Networks for a Flatland Agent - Project Report}} % Article title

\author{
    \large
    \textsc{Mathias Ose \& Øyvind Robertsen} \\ % Your name
    \normalsize Norwegian University of Science \& Technology \\ % Your institution
    \normalsize \href{mailto:mathiabo@stud.ntnu.no}{mathiabo@stud.ntnu.no}, \href{mailto:oyvinrob@stud.ntnu.no}{oyvinrob@stud.ntnu.no} % Your email address
    \vspace{-5mm}
}
\date{}

%----------------------------------------------------------------------------------------

\begin{document}

\maketitle % Insert title

\thispagestyle{fancy} % All pages have headers and footers

%----------------------------------------------------------------------------------------
%	ABSTRACT
%----------------------------------------------------------------------------------------

\begin{abstract}

\noindent This report describes a solution to Project 3 in the subject IT3708 at NTNU. 
The purpose of this project is to use an evolutionary algorithm to tune the weights of an artificial neural network, using the networks performance as an agent in a 2D world with simple rules as a fitness measure.
\end{abstract}

%----------------------------------------------------------------------------------------
%	ARTICLE CONTENTS
%----------------------------------------------------------------------------------------

\begin{multicols}{2} % Two-column layout throughout the main article text

  \section{EA design}

  \subsection{EA parameters}

  Table~\ref{tbl:ea-parameters} gives an overview of the parameters with which we achieved the best performance.

  \begin{table}[H]
    \begin{tabular}{|l|l|}
      \hline
      Population size                   & 200                  \\ \hline
      Generations                       & 100                  \\ \hline
      Crossover rate                    & 0.5                  \\ \hline
      Mutation rate                     & 0.01                 \\ \hline
      Adult selection                   & Generational mixing  \\ \hline
      Adult to child ratio              & 0.5                  \\ \hline
      Parent selection                  & Tournament selection \\ \hline
      Tournament selection bracket size & 8                    \\ \hline
      Epsilon                           & 0.15                 \\ \hline
      Crossover operator                & One point crossover  \\ \hline
      Mutation operator                 & Per genome component \\ \hline
    \end{tabular}
    \caption{Table of EA parameters}
    \label{tbl:ea-parameters}
  \end{table}

  \subsection{Fitness function}

  \begin{gather*}
    s_{food} = \frac{f_{eaten}}{f_{total}} \\[10pt]
    s_{poison} = \frac{p_{eaten}}{p_{total}} \\[10pt]
    f = s_{food} - s_{poison}
  \end{gather*}

  The equations above describe the fitness function we implemented.
  The fitness $f$ is equal to the penalty for poison eaten $s_{poison}$ subtracted from the score rewarded for food eaten $s_{food}$.
  The food reward and the poison penalty are the shares of total food/poison the agent has consumed.
  In other words, a ``perfect'' agent is one that eats all the food on a board instance and avoids all poison.

  \section{ANN implementation}

  \subsection{ANN design}

  \def\layersep{2.0cm}
  

  \begin{figure}[H]
    \centering
    \begin{tikzpicture}[shorten >=1pt,->,draw=black!50, node distance=\layersep]
      \tikzstyle{every pin edge}=[<-,shorten <=1pt]
      \tikzstyle{neuron}=[circle,fill=black!25,minimum size=17pt,inner sep=0pt]
      \tikzstyle{input neuron}=[neuron, fill=green!50];
      \tikzstyle{output neuron}=[neuron, fill=red!50];
      \tikzstyle{hidden neuron}=[neuron, fill=blue!50];
      \tikzstyle{annot} = [text width=4em, text centered]

      % Draw the input layer nodes
      \foreach \name / \y in {1,...,6}
      % This is the same as writing \foreach \name / \y in {1/1,2/2,3/3,4/4}
      \node[input neuron, pin=left:Input \#\y] (I-\name) at (0,-\y) {};

      % Draw bias node
      \node[neuron, pin={[pin edge={<-}]above:Bias}, right of=I-1] (B) {};
      
      % Draw the hidden layer nodes
      \foreach \name / \y in {1,...,3}
      \path node[output neuron, pin={[pin edge={->}]right:Output \#\y}, right of=O-\name] (O-\name) at (\layersep,-1.75*\y cm) {};

      % Draw the output layer node
      %\node[output neuron,pin={[pin edge={->}]right:Output}, right of=I-3] (O) {};

      % Connect every node in the input layer with every node in the
      % hidden layer.
      \foreach \source in {1,...,6}
      \foreach \dest in {1,...,3}
      \path (I-\source) edge (O-\dest);

      % Connect the bias node to all output nodes
      \foreach \dest in {1,...,3}
      \path (B) edge (O-\dest);

      % Connect every node in the input layer with the output layer
      %\foreach \source in {1,...,5}
      %\path (I-\source) edge (O-);

      % Annotate the layers
      %\node[annot,above of=H-1, node distance=1cm] (hl) {Hidden layer};
      \node[annot,above of=O-1] (Ao) {Output layer};
      \node[annot, left of=Ao, node distance=4cm] {Input layer};
    \end{tikzpicture}

    \caption{ANN layout} \label{fig:ann-layout}
  \end{figure}

  Figure~\ref{fig:ann-layout} shows the ANN design with which we achieved the best results with.
  There are six input neurons, three representing the food sensors and three representing the poison sensors.
  Each input neuron is connected to each of the output neurons.
  The three outputs control the agents movement.
  We used a single bias node, outputting $1.0$ to the output layer.

  We used a standard sigmoid for our activation function.
  To some initial surprise on our part, we achieved the most consistent results using binary weights.

  \subsection{Design process}

  As we had no prior experience in designing ANNs, our approach was slightly unguided.
  We started out using 8-bit weights, and a single hidden layer in addition to the design depicted in figure~\ref{fig:ann-layout}.
  Through several evolutionary runs, with varying parameters used for the runs (increasing selection pressure), the best individuals using this design achieved fitness values between $0.35$ and $0.6$.
  Note that the codomain of our fitness function is $[-1, 1]$.
  While these values don't m, we noticed that these agents decreased their score by making stupid choices, not by being placed in situations in which they would have no other choice than to eat poison.

  As the desired agent functionality is very simple, we started gradually simplifying our network.
  Removing the hidden layer yielded positive results, so on a whim, we decided to run a simulation with binary weights.
  (Encoding weights as 1-bit integers in our genomes that is.)
  We were surprised at first, as our first run with this weight-encoding was also the first run in which an agent achieved a fitness of $1.0$.

  However, after giving it some thought, we realised that this design restricted the search space for our EA, while still allowing for an optimal agent function to be encoded in the weights.


  \section{EA performance}

  \subsection{Static, single scenario}

  \begin{figure}[H]
    \centering
    \includegraphics[width=\linewidth]{images/static_1.png}
    \caption{Fitness plot for static, single scenario run.} \label{fig:static-single}
  \end{figure}
  
  \subsection{Static, five scenarios}

  \begin{figure}[H]
    \centering
    \includegraphics[width=\linewidth]{images/static_5.png}
    \caption{Fitness plot for static, five scenario per generation run.} \label{fig:static-single}
  \end{figure}

  \begin{figure}[H]
    \centering
    \includegraphics[width=\linewidth]{images/dynamic_1.png}
    \caption{Fitness plot for dynamic, single scenario run.} \label{fig:static-single}
  \end{figure}

  \begin{figure}[H]
    \centering
    \includegraphics[width=\linewidth]{images/dynamic_5.png}
    \caption{Fitness plot for dynamic, five scenarios per generation.} \label{fig:static-single}
  \end{figure}

\end{multicols}

%\bibliography{references}
%\bibliographystyle{plain}

\end{document}
